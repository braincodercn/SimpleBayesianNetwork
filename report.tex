\documentclass{article}

\usepackage{amsmath}
\usepackage{listings}
\usepackage{xcolor}
\lstset { %
    language=C++,
    backgroundcolor=\color{black!5}, % set backgroundcolor
    basicstyle=\footnotesize,% basic font setting
}

\title{Knowledge Graph Based on Bayesian Network}
\date{2018-9-8}
\author{Lanting Guo}

\begin{document}
\maketitle
\newpage
\tableofcontents
\newpage
\pagenumbering{gobble}

\section{Why Bayesian Network}
To empower both data and human knowledge.
\section{Definitions}
\subsection{Node}
\begin{lstlisting}
  using Attributes = Dict<AttributeName, Value>;
  class Node {
    NodeName   node_name;
    Node       parent;
    Nodes      childeren;
    Attributes attributes;
  };
\end{lstlisting}
To represent Question, Person, Concept etc;
\subsection{Edge}
\begin{lstlisting}
  class Edge {
    Node head;
    Node tail;
  };
\end{lstlisting}

\section{Goals}
\subsection{students' perspective}
philosophy: Accelate learning speed.
definition: At Time t1, Student s understands Concept c with a Probality p1, after a fixed period training at platform,
            at Time t2, Student s understands Concept c with a Probality p2,

details: 
\subsection{company's perspective}
\section{Graph Architecture}
\subsection{baisc definition}
\subsection{how to add a node}
\subsection{how to delete a node}
\subsection{how to add a parent node}
\subsection{how to add a child node}

\section{Evaluation, Metrics}

\section{Inference}

\section{Parameters: Learning, Hand Coding}

\section{Scalability}

\section{Conclusion}

\end{document}
